
\section{Team Member's Roles}
%Describe who was involved and what role(s) were played. 

Product development is divided into two teams:
\begin{enumerate}
    \item Web Team and Database:
        \begin{itemize}
            \item Daniel Hodgin
            \item Brady Shimp (Scrum Master)
            \item Savoy Schuler
        \end{itemize}
    \item Conversion Software and Mobile Application Team:
        \begin{itemize}
            \item Aaron Alphonsus
            \item Cheldon Coughlen (Team Lead)
            \item Kenneth Petry
        \end{itemize}
\end{enumerate}

The website team shares the responsibility for developing the website, file upload and download abilities, an API for connecting the website to the conversion software, user log in protected profile functionality, user abilities to manage files, file permissions and cloud hosting abilities.  


The conversion software shares the responsibility for developing software to convert uploaded file types into file types render-able by AR devices and applications needed by devices for reading and rendering files from QR codes.  Additionally,  this sub-team is responsible for producing a mobile application that interfaces with the website to provide AR visualization.


As Team Lead, Cheldon Coughlen acts as the team representative to the sponsor and client and brokers communication between these parties and the development team. 

As Scrum Master, Brady Shimp manages the task board and delegates tasks. 

\section{Project  Management Approach}
%This section will provide an explanation of the basic approach to managing the project.  Typically, this would detail how the project will be managed through a given Agile methodology.  The sprint length (i.e. 2 weeks) and product backlog ownership and location (ex. Trello) are examples of what will be discussed.  An overview of the system used to track sprint tasks, bug or trouble tickets, and user stories would be warranted. 

The product is being approached with Agile methodology and two week sprints. InTouch L.L.C. COO Brady Shimp owns the backlog which is located on the GitHub project repository. Mr. Shimp creates tickets which are placed in the backlog. Developers will select tickets, attach their name to it, and move it to an "In Progress" bin to denote activity. Tickets may be assigned by Mr. Shimp or selected by unoccupied developers. Priority levels are assigned to tasks, bugs, and user stories to indicate the priority of the respective ticket. These priority levels may be assessed by whether the ticket roadblocks other development, necessity of the feature based on milestones, or urgency otherwise established.

\section{ Stakeholder Information}


% This section would provide the basic description of all of the stakeholders for 
% the project. Who has an interest in the successful and/or unsuccessful completion of this project? 

Following are descriptions and interests of key stakeholders in the development of Augmented Education:

\begin{itemize}
	\item InTouch L.L.C.: Custom software solutions company sponsoring the project on behalf of the South Dakota School of Mines and Technology. 
	\item South Dakota School of Mines and Technology: Higher education STEM university seeking to integrate mixed reality with STEM education and develop curricular content.
	\item Dr. Jeff McGough: Computer science and mathematics professor seeking to develop mixed reality curricular content for the South Dakota School of Mines and Technology. 
	\item Dr. Adam Piper: Industrial engineering professor requesting to use Augmented Education to simulate design labs.
	\item Dr. Brent Deschamp: Mathematics professor requesting to use Augmented Education for the purpose of displaying three dimensional mathematical models in the classroom. 
	\item Dr. King Adkins: Humanities professor seeking to integrate Augmented Education into humanities classes. 
	\item Dr. Christer Karlsson: Computer science professor seeking to integrating mixed reality into computer science education.
\end{itemize}


\subsection{Customer or End User (Product Owner)}
% Who?  What role will they play in the project?  Will this person or group manage 
% and prioritize the product backlog?  Who will they interact with on the team to 
% drive product backlog priorities if not done directly? 

The customer is the South Dakota School of Mines and Technology with faculty members such as Dr. Jeff McGough, Dr. Christer Karlsson. Dr. Adam Piper, Dr. Brent Deschamp, and Dr. King Adkins constituting an initial base of end users. These faculty members are directly involved with the creative direction of the product and are the indirect source of the product backlog. The faculty members meet monthly with the development team to develop user stories. Two faculty members, Dr. Jeff McGough and Dr. Christer Karlsson, meet weekly with the development team to provide direct input on the backlog and direction of the product. 

\subsection{Management or Instructor (Scrum Master)}
% Who?  What role will they play in the project?  Will the Scrum Master drive the 
% Sprint Meetings? 

Brady Shimp, COO of InTouch L.L.C., is the Scrum Master of this project. Brady developed the project time-lines and milestones and actively manages the task board and drives delegation. Brady leads sprint meetings and stand-ups. 

\subsection{Developers --Testers}
% Who?  Is there a defined project manager, developer, tester, designer, architect, 
% etc.? 

Developers are assigned components of the platform on constant rotation. Each developer is required to play all roles regarding their component. This includes, but is not limited to, designing, architecting, managing, developing, and testing their solution. In the designing and architecting phase, a developer is required to consult the members of their development team to ensure proper planning and compatibility. Each developer must have their functioning implementation thoroughly verified by another team member before merging their solution into the development branch. 

As a whole, the project development and testing is managed by Brady Shimp. Mr. Shimp manages the development of the project through the task board by creating, delegating, and monitoring the completion of tickets. 

\section{Budget}
% Describe the budget for the project including gifted equipment and salaries for 
% people on the project.

This project received \$15,500 of funding from the South Dakota School of Mines and Technology's Mobile Computing Grant. \$9000 was allocated for student developer salary. \$6,500 was used for hardware purchases which included a Microsoft HoloLens, Meta Meta 2, Oculus Rift, Samsung Galaxy S8, Google Pixel, and certain computer hardware necessary for running the Oculus Rift and Meta 2. 
 
The South Dakota School of Mines and Technology received the Mobile Computing Grant for the purpose of developing mixed reality courses and curriculum content. The grant proposal may be found in F-\ref{F:F-1}.

\section{Intellectual Property and Licensing}
% Describe the IP ownership and issues surrounding IP.

All intellectual property and ownership is retained by the South Dakota School of Mines and Technology.

Developer licenses for Visual Studio 2017 are provided by the South Dakota School of Mines and Technology. Android Studio is license free. All other third-party resources utilized are available under MIT License. 

