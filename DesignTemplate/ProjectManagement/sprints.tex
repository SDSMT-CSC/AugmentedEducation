
\section{Sprint Schedule}
As stated above, the sprint cycles for Senior Design I were to be two weeks long. Each sprint was designed to accomplish a milestone, or set of related tasks. Below is the schedule of the sprints and what was accomplished during that time.

\begin{itemize}
	\item 9/18/2017   - Define milestones and user stories.
	\item 10/2/2017   - Base website created with file upload ability.
	\item 10/16/2017  - Wire frames, Presentation 1 documents, standalone file conversion.
	\item 10/30/2017  - File upload/download, file conversion integrated.
	\item 11/13/2017  - QR code generation, standalone advanced file conversion.
	\item 11/27/2017  - File conversion integration, create users, Presentation 2 documents.
	\item 12/04/2017  - Finish documentation, second semester preparation.
	\item 12/13/2017  - MVP delivered to sponsor.
	\item 04/26/2018  - Final product delivered to sponsor.
\end{itemize}

As stated above, the Senior Design II project management methodology switched from an Agile methodology to a Feature Driven Development methodology.  
This no longer held a rigid schedule for task deadlines and frequent meetings.
Instead, tasks were simply posted and worked on in a more free-form manner until they were completed. 
The tasks for Senior Design II were as follows:

	\begin{itemize}
		\item Website:
		\begin{itemize}
			\item Switch from the Azure Student free trial account to a Pay-As-You-Go subscription.
			\item Fully integrate the QR Code generator throughout the web application.
			\item Implement Microsoft Identity and Entity Framework as the application user management and DataBase providers.
			\item User profile creation and management through the web application interface.
			\item Implement Azure Blob Storage and structure patterns for user cloud file storage.
			\item Integrate web services with the mobile Android application.
			\item Explore web application security options (TLS).
			\item Refine the User Interface.
			\item Testing and documentation.
		\end{itemize}

		\item HoloLens:
		\begin{itemize}
			\item Create a custom Unity application capable of the following:
				\begin{itemize}
					\item Have an interactive user interface capable of utilizing common HoloLens gesture commands.
					\item Scan QR codes within the application.
					\item Retrieve 3D files from the web application by scanning QR codes.
					\item Render 3D models into the Unity application at runtime by scanning QR codes.
					\item Allow users to manipulate the rendered models in real-time.
				\end{itemize}
			\item Development on this application was abandoned due to an update to the HoloLens default 3D viewer application that covered all previously listed requirements.
		\end{itemize}

		\item Mobile Application:
		\begin{itemize}
			\item Determine the development platform and technology.
			\item Begin the application with ARCore package integration.
			\item Request and set appropriate user permissions on the device.
			\item Download 3D files by scanning QR codes within the application.
			\item Website integration:
			\begin{itemize}
				\item User login and authentication token retrieval.
				\item Listing files that the authenticated user has access to.
				\item Allowing selective download of available files without a QR code.
			\end{itemize}
			\item Refine the user interface.
			\item Testing and documentation.
		\end{itemize}
	\end{itemize}

\section{Timeline}
Below is the timeline of sprint milestones accomplished in the first semester. Due to the switch to FFD, there is not a timeline for the second semester.

\begin{figure}[h]
\begin{center}
\includegraphics[width=1\textwidth]{./SprintGanattChart}
\end{center}
\caption{Sprint Timeline for the First Semester}
\end{figure}
