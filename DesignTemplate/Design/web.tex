
 \section{Website}

 \subsection{Overview}
 \paragraph{}
 A major tool in the project is the website interface for hosting the 3D models that users generate.
 The website is where a user will be able to upload their 3D models
 and get the reference tag for that model.
 
 \paragraph{}
 The website is intended to be a full user experience, where they can
 can sign in to view and manage their own content. In the long term view
 this will be a service similar to GitHub, where users can collaberate and share
 their material with friends and co-workers.

\paragraph{}
This website will be making use of Microsoft's ASP.NET stack along with
Azure services for database and website hosting. A more indepth look into 
these tool will be offered below.

\subsection{User Interface}
The user interface is made to be very simple. The final product
UI was decided to be made of five main windows. 
\begin{itemize}
    \item Home Screen
    \begin{itemize}
        \item Here the user will be able to navigate to the different sections of the website.
        \item The user will need to be logged in to navigate to their pages.
        \item It has a link link to the sign in and sign up.
        \item If the user tries to navigate to a page without signing in
        they will be taken to the login page.
    \end{itemize}

    \item Upload Screen
    \begin{itemize}
        \item This is a very simple page where the user can upload their files
        \item Once the file is uploaded a QR code will be showing for a reference to the file.
    \end{itemize}

    \item User, Organization and All Screens
    \begin{itemize}
        \item These screens will all be very similar.
        \item They will allow for the user to search through all
        available models associatated to the section they are in.
    \end{itemize}
\end{itemize}

The wireframes of all of these pages can be found in prototypes and demos section of this document.
You can find refeneces to the wireframes below.
\begin{itemize}
    \item Home Page ~\ref{fig:proto_web_home}
    \item User Page ~\ref{fig:proto_web_user_page}
    \item Organization Page ~\ref{fig:proto_web_organization_home}
    \item Browse All Page ~\ref{fig:proto_web_browse_all_files}
    \item Upload Page ~\ref{fig:proto_web_upload}
\end{itemize}

\subsection{Technologies Used}

    As stated earlier in this document, this product is making use
    of the Microsoft environment for its tools. Below is an indepth 
    break down of the tools currently being used:

    \begin{itemize}
    \item ASP.NET MVC
    \begin{itemize}
        \item An MVC web architecture, where the backend logic is writen in controller classes
        that send and recieve data from the client.
        \item It allows for dynamic html pages using a Razor synatax. Razor allows
        you to embed C\# Code into the html and execute logic.
    \end{itemize}
    
    \item Azure
        \begin{itemize}
            \item Microsoft cloud hosting services.
            \item Allows for simple database and web hosting.
            \item Paid features are offered free to students.
        \end{itemize}
    \end{itemize}

    These tools make up the core development of this website. The website is also
    making use of other smaller packages to handles user authentication and database management.

    The website also impliments the file conversion software that is described later.
    The file conversion was written in C++, so it couldn't be compiled with the website.
    The work around was that was used, was the file conversion was compiled into a executable.
    The executable was used through a system command to convert the file.


    \subsection{Data Flow}
    The data flow happens in two main places. It is set up into a Upload controller
    and a Download Controller.

    \begin{itemize}
        \item Upload Controller
        \begin{itemize}
            \item The user browses through the their file system and uploads
            the 3D model that selected.
            \item Once the file is on the server side it is then converted to an
            .fbx and stored for download.
            \item A tag to reference the model is returned to the client.
        \end{itemize}
        
        \item Download Controller
        \begin{itemize}
            \item The client sends the reference tag was given when it was uploaded to the server.
            \item The server takes that tag and finds the assoicated model.
            \item Using the httpresponse datastructure to create the response and send
            the model back to the user.
        \end{itemize}
    \end{itemize}


\subsection{Design Details}

    \subsubsection{Overview}

    The website is written in C\# with the ASP.NET MVC framework. The main logic
    of the website is contained in two controllers
    \begin{itemize}
        \item Upload Controller
        \begin{itemize}
            \item Handles upload from client to the server.
            \item Uses File Conversion executable to convert to .fbx
            \item Stores file on the web.
        \end{itemize}
        \item Download
        \begin{itemize}
            \item Uploads Model tag from client to server.
            \item Finds associated store model.
            \item Returns model to the client.
        \end{itemize}
    \end{itemize}

    \subsubsection{Code Structure}
    As stated before the website is using ASP.NET MVC framework. 
    The logic happens in the controllers and is split up into the upload and download controller.
    
    \paragraph{Parameter Passing}
    \hfill \break
    The main functions that are being used are the upload and download function.
    
    \begin{itemize}
        \item Upload Function
        \begin{itemize}
           \item HttpPostedFileBase - http message sent from client to server containing the file
           \item The message is parsed to retrieve the file and save to the web.
        \end{itemize}

        \item Download Function
        \begin{itemize}
            \item input string ModelTag - string containing the reference tag to the model.
            \item output httpresponse - http message sent from server to client containing the file. 
        \end{itemize}
    \end{itemize}