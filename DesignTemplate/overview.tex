% !TEX root = DesignDocument.tex


\chapter{Overview, Description and Deliverables}

% The overview should take the form of an executive summary.  Give the reader a feel 
% for the purpose of the document, what is contained in the document, and an idea 
% of the purpose for the system or product. 

\section{Team Members and Team Name}

Team Name: Augmented Education

\noindent Team Members:
\begin{itemize}
	\item Aaron Alphonsus
	\item Cheldon Coughlen
	\item Daniel Hodgin
	\item Kenneth Petry
	\item Savoy Schuler
	\item Brady Shimp
\end{itemize}

\section{Client}
% A description of the client or customer.
% Description of sponsor if different than client.
% Brief statement of customer's problem or goal for this project
% List the customer needs

The project client is the South Dakota School of Mines and Technology (SD Mines) which will be purchasing a one-year Early Adopter License for the product. The client is intending to use the product to enhance the traditional education experience by integrating media unique to augmented reality in the classroom. The client needs a platform through which three-dimensional computer aided design files may be uploaded, cloud hosted, and delivered to an augmented reality device for rendering via a QR code associated with the uploaded file. The client will also need to be able to manage files and would like a collaboration space available for public and private sharing of files. 

The project sponsor is InTouch L.L.C., a custom software solutions provider specializing in mixed reality software. The sponsor needs the product to adequately meet the expectations set forth by its contract with the client. 

\section{Product}
% A high level description of the project.  Project environment ...
% Project boundaries.
% Project context.
% Technical Environment.
% Current systems overview.


\subsection{Description}
Augmented Education is a platform allowing three-dimensional models made with common computer-aided design (CAD) programs to be cloud hosted, converted, and made on-demand available to augmented reality viewing devices and mobile phones through download links such as QR codes that may be embedded in textbooks, presentations, and other media to enhance the traditional classroom experience. Once a 3D design has been accessed via QR code and rendered, the student may manipulate the rendering as supported by individual AR devices. 

\subsection{Product Vision}
Long term product evolution is visualized in the following phases. The product will be delivered to end users upon completion of the first phase. The licensing model will allow the following two phases to commence with user feedback and active revenue streams. 

\begin{itemize}
\item Phase One - current development - QR code and multi-platform visualization.
\item Phase Two - social platform to share models and collaboration both within universities and across other educational institutions.
\item Phase Three - full blown platform with API ecosystem that connects all sorts of data from various applications to visualize and collaborate.
\end{itemize}

\subsection{Phase One Features}
\begin{itemize}
	\item Users may use the website interface to upload files from a CAD program commonly used in STEM education. 
	\item  Upon upload files will be converted into a ubiquitous file format available for AR rendering. 
	\item The user will be returned a QR code that may be embedded in textbooks, homeworks handouts, PowerPoint presentations, emails, etc. 
	\item When an AR headset or mobile phone is used to view the QR code, the device will locate and render the associated file in augmented reality for the user.  
	\item Certain devices will allow the user motion control abilities to interact with the renderings so that they may be moved, scaled, rotated, animated, or “flipped through” in steps. 
	\item Cloud hosting will make the user’s files available anywhere at any time. 
	\item The web interface will allow for the management of files (add, delete, update, download). 
\end{itemize}

\subsection{Phase Two Features}
\begin{itemize}
	\item Web interface privacy settings will allow designs to be private, public, or accessible only by a “group” such as an institution. 
	\item A page will be added for showcasing popular public designs. 
	\item The public design page will have filtering abilities for sorting designs by upload date, most viewed, and by time periods. 
	\item Users will have a personal feed showcasing designs from collaboration groups. 
\end{itemize}

\subsection{Phase Three Features}
\begin{itemize}
	\item Expand platform to support broad range of file types for the purpose of increasing the number of CAD programs supported. 
	\item Expand platform to support broad range of hardware for the purpose of increasing number of AR devices supported. This may include adding support for output file types or application development for hardware. 
\end{itemize}


\subsection{Intellectual Property}
Patent application for elements of process pending. 

\subsection{Value} 
A/B testing targeting student engagement, retention, and conceptual clarity will be conducted in classrooms at the South Dakota School of Mines and Technology. Tests will be performed for several semesters wherein one section of a course taught by a given professor will be able to utilize our technology in the classroom as a learning aid and the other will not. 

The time required for an engineer to be trained in the transition between student and working professional averages in the range of 1-2 years. A result of the goal of enhancing the traditional educational experience with this service is to eliminate a large portion of the training required for students to make the transition to professionals in their fields. The test results will be able to support the claim that this service will reduce the amount of time is required to successfully make this transition.  With the average entry level engineering salary at about \$70,500, meaning up to \$141,000 or more in training expenses per entry level hire, Augmented Education aims to cut this need in half by providing students with a more immersive approach to learning and mastering concepts of design and structure. This is envisioned through noting that, should the new in-classroom experience create more effective learning, more topics and depth will be able to be taught per course. 

\subsection{Value Testimonials}
“Education is an industry based on entertainment. You can learn everything about Calculus from a 1960’s textbook. The information is the same today as it was then, but new textbooks are sold because they are printed in color and with better pictures. Students use Youtube tutorials to learn math because it is effective. The reason people pay for classes with instructors is because we balance presenting information with entertaining the student’s interest in it. Augment reality in education is that next step in creating a more engaging and entertaining learning environment.” - Dr. Jeff McGough, Computer Science and Mathematics Professor at SDSMT and founder of InTouch L.L.C.

“As an industrial engineering instructor, I run labs where students build bridges and motors using Lego bricks. I do it because it's interactive and helps communicate some of the early concepts. If I could  have system where students could instead see the components of these structures in an animated 3D environment that they could interact with, I would implement it immediately.” - Dr. Adam Piper, Industrial Engineer Professor at SDSMT



\subsection{Mission Statement}
InTouch L.L.C. pursues the mission of developing augmented reality and virtual reality (known together as “mixed reality”) solutions for education and enterprise. Hardware and entertainment software for this technology have matured over the past decade while innovators have, until now, overlooked the opportunity to leverage this same technology for applications such as classroom education, 3D advertising, architecture design, pre-construction modeling, and more.  

\subsection{Elevator Pitch}
Augmented reality, commonly called AR, is a technological advancement that allows individuals to overlay virtual animations into the real world using an optical viewing aid to augment the user’s vision of their surroundings. Though hardware and entertainment software is blossoming, this cutting edge technology has yet to penetrate the eager and profitable industry of higher education. Augmented Education by InTouch L.L.C. provides a platform for harnessing this new technology to enhance the traditional education experience with an effective and engaging new medium. By orienting itself at instructors and students, Augmented Education is a cloud hosted service that opens up multiple channels of revenue and sets in place an infrastructure and a connection through which numerous value-added services can be provided at the user’s pleasure.

To demonstrate the Augmented Education service, think back to the last time you were in class learning about a 3D design, Calculus graph, or physics problem. No instructor had a choice other than to present 3D content on a 2D chalkboard or projector. Imagine next year you sit in  a South Dakota School of Mines and Technology classroom where the instructor asks you to use your phone or a headset to view their presentation. What was once a QR code in the presentation is now a 3D shape appearing in the environment with you. Using your hands, you may bring it closer, manipulate it, turn it around, and perhaps flip through a sequence of animations using your fingertips.

Architecture and civil engineering students often design structures and buildings with 3D design software. With this platform, a student or instructor need only to upload their file to the Augmented Education website before they are able to use an augmented reality headset to scale their design to real world size and step through it, viewing it from the inside or placing it next to a campus building for scale. As collaboration grows, these students may soon be able to use this platform to virtually inspect the architecture of famous buildings from around the world. 

The first area Augmented Education is going to hit is the South Dakota School of Mines and Technology which has already purchased a one-year license for Augmented Education. This also is where Augmented Education is currently being tested for results in student engagement, retention, and conceptual clarity. Augmented Education intends to spread to textbook companies and other STEM programs around the Midwest by partnering in sales with the 3D modeling software companies that are most widely used in the STEM community. STEM programs are an excellent starting point to find early adopters like SD Mines due to their intrinsic need to stay at the forefront of technology and innovation. 

Augmented Education will offer different tiers of licensing with increasing cloud storage and value-add services for each, with a basic license starting around \$7,000 for one terabyte private storage and five terabytes downstream bandwidth, DDoS protection, and load balancing. 


\subsection{Purpose of the System}
% What is the purpose of the system or product? 

The purpose of this product is to enhance the value of CAD software common to STEM programs and provide a higher quality education by giving students the ability to view CAD visualizations in a true 3D environment allowing students to fully perceive depth, scale, volume, and attributes through object manipulation features.


\section{Business/Market Need}
% Use this section to define what business need exist and how this software will 
% meet and/or exceed that business need.    How do you make money!!  What is the % revenue model?  What is the market? Who are customers?

% \noindent
% \underline{Example:  Mouse Detector Phone App}

% \begin{description}
% \item [Product Description:] iPhone based app that can detect the high frequency sounds of mice and locate them.

%  \item [Key Business Goals:] Product introduced in the second quarter 2009
% \begin{itemize}
% \item 50\% gross margin
% \item 15\% share of mouse trap market
% \end{itemize}

% \item [Primary Market:] Consumers
% \item [Secondary Markets:] Lazy cats

% \item [Assumptions:]  ~~ \\
% \begin{itemize}
% \item Available from App store
% \item Survillence mode
% \item Low power consumption
% \item Autodial on detection
% \end{itemize}

% \item [Stakeholders:]  ~~ \\
% \begin{itemize}
% \item User
% \item Retailer
% \item Sales Force
% \item Production
% \item Legal department
% \end{itemize}

% \item [Certifications:] Apple, Cat Fancy Magazine
% \end{description}

\begin{description}
	\item [Product Description:] AR CAD visualization platform.
	
	\item [Key Business Goals:] Product introduced in the second quarter 2018.
	\begin{itemize}
		\item 40\% gross margin
		\item 80\% share of CAD to AR education market
	\end{itemize}
	
	\item [Primary Market:] CAD software distributors
	\item [Secondary Markets:] Textbook publishers
	\item [Secondary Markets:] Higher education institutions
	
	\item [Assumptions:]  ~~ \\
	\begin{itemize}
		\item Platform integrates with AR devices 
		\item Platform accepts file formats from wide range of CAD programs
		\item Higher education institutions invest in AR technologies
	\end{itemize}
	
	\item [Stakeholders:]  ~~ \\
	\begin{itemize}
		\item Users (Faculty)
		\item Users (Students)
		\item Department
		\item Institution
		\item Software Distributor
		\item Textbook Publisher
	\end{itemize}
	
	\item [Certifications:] South Dakota School of Mines and Technology
\end{description}

\section{Deliverables}

%Provide a complete description of the client requested deliverables. This section should be the section that your software contract refers to. (e.g. prototype, documentation, code, users manual, ...)
 

\subsection{Software}
The sponsor deliverable is a software tool chain to save, retrieve, and view 3D models produced in popular modeling software. The two main components are:

\begin{enumerate}
	\item A website to manage users' files
		\begin{itemize}
			\item Upload files
			\item Save files
			\item Run software to convert between 3D file types
			\item Serve files back to users
		\end{itemize}
	\item A file conversion program to convert a users uploaded file into a viewable file type
		\begin{itemize}
			\item Convert a given 3D model into a common file type to be stored on the website
			\item Convert the common file type to the type needed to be viewd on an Augmented Reality device
		\end{itemize}
\end{enumerate}

\noindent The client deliverable is a One-Year Early Adopter License for the platform that operates with:

\begin{itemize} 
	\item 1 terabyte private storage 
	\item 5 terabytes downstream bandwidth (per month)
	\item DDoS protection
	\item load balancing
\end{itemize} 

\subsection{Hardware}

Test the flow of the website and file conversion software on popular Augmented Reality devices, which may include:
\begin{itemize}
	\item Microsoft Hololens
	\item Meta 2
	\item Mobile devices running IOS and/or Android
\end{itemize}

\subsection{Documentation}
%And so on.  Anything that your contract states that you will deliver to the client.

The sponsor will be delivered product documentation for the purpose of further feature development. The client will be provided a product User Manual. 
