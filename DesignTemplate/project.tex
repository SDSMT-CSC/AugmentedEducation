% !TEX root = DesignDocument.tex

\chapter{Project Management}

\hspace{7mm}
This chapter encompasses information such as to how the development of the project will be structured
by defining roles and responsibilities of those involved, the development process and environment,
requirements, terms and conditions, and scheduling.

\section{Team Member's Roles}
%Describe who was involved and what role(s) were played. 

Product development is divided into two teams:
\begin{enumerate}
    \item Web Team:
        \begin{itemize}
            \item Daniel Hodgin
            \item Brady Shimp (Scrum Master)
            \item Savoy Schuler
        \end{itemize}
    \item Conversion Software Team:
        \begin{itemize}
            \item Aaron Alphonsus
            \item Cheldon Coughlen (Team Lead)
            \item Kenneth Petry
        \end{itemize}
\end{enumerate}

The website team shares the responsibility for developing the website, file upload and download abilities, an API for connecting the website to the conversion software, user log in protected profile functionality, user abilities to manage files, social/collaborative features, file permissions and cloud hosting abilities.  


The conversion software shares the responsibility for developing software to convert uploaded file types into file types render-able by AR devices and applications needed by devices for reading and rendering files from QR codes.


As Team Lead, Cheldon Coughlen acts as the team representative to the sponsor and client and brokers communication between these parties and the development team. 

As Scrum Master, Brady Shimp manages the task board and delegates tasks. 

\section{Project  Management Approach}
%This section will provide an explanation of the basic approach to managing the project.  Typically, this would detail how the project will be managed through a given Agile methodology.  The sprint length (i.e. 2 weeks) and product backlog ownership and location (ex. Trello) are examples of what will be discussed.  An overview of the system used to track sprint tasks, bug or trouble tickets, and user stories would be warranted. 

The product is being approached with Agile methodology and two week sprints. InTouch L.L.C. COO Brady Shimp owns the backlog which is located on the GitHub project repository. Mr. Shimp creates tickets which are placed in the backlog. Developers will select tickets, attach their name to it, and move it to an "In Progress" bin to denote activity. Tickets may be assigned by Mr. Shimp or selected by unoccupied developers. Priority levels are assigned to tasks, bugs, and user stories to indicate the priority of the respective ticket. These priority levels may be assessed by whether the ticket roadblocks other development, necessity of the feature based on milestones, or urgency otherwise established.

\section{ Stakeholder Information}


% This section would provide the basic description of all of the stakeholders for 
% the project. Who has an interest in the successful and/or unsuccessful completion of this project? 

Following are descriptions and interests of key stakeholders in the development of Augmented Education:

\begin{itemize}
	\item InTouch L.L.C.: Custom software solutions company needs product to fulfill contractual obligations to client. 
	\item South Dakota School of Mines and Technology: Higher education STEM university seeking to integrate mixed reality with STEM education and develop curricular content.
	\item Dr. Jeff McGough: Computer science and mathematics professor seeking to develop mixed reality curricular content for the South Dakota School of Mines and Technology. 
	\item Dr. Adam Piper: Industrial engineering professor requesting to use Augmented Education to simulate design labs.
	\item Dr. Brent Deschamp: Mathematics professor requesting to use Augmented Education for the purpose of displaying three dimensional mathematical models in the classroom. 
	\item Dr. King Adkins: Humanities professor seeking to integrate Augmented Education into humanities classes. 
	\item Dr. Christer Karlsson: Computer science professor seeking to integrating mixed reality into computer science education.
\end{itemize}


\subsection{Customer or End User (Product Owner)}
% Who?  What role will they play in the project?  Will this person or group manage 
% and prioritize the product backlog?  Who will they interact with on the team to 
% drive product backlog priorities if not done directly? 

The direct customer is the South Dakota School of Mines and Technology with faculty members such as Dr. Jeff McGough, Dr. Christer Karlsson. Dr. Adam Piper, Dr. Brent Deschamp, and Dr. King Adkins constituting an initial base of end users. These faculty members are directly involved with the creative direction of the product and are the indirect source of the product backlog. The faculty members meet monthly with the development team to develop user stories. Two faculty members, Dr. Jeff McGough and Dr. Christer Karlsson, meet weekly with the development team to provide direct input on the backlog and direction of the product. 

\subsection{Management or Instructor (Scrum Master)}
% Who?  What role will they play in the project?  Will the Scrum Master drive the 
% Sprint Meetings? 

Brady Shimp, COO of InTouch L.L.C., is the Scrum Master of this project. Brady developed the project time-lines and milestones and actively manages the task board and drives delegation. Brady leads sprint meetings and stand-ups. 

\subsection{Developers --Testers}
% Who?  Is there a defined project manager, developer, tester, designer, architect, 
% etc.? 

Developers are assigned components of the platform on constant rotation. Each developer is required to play all roles regarding their component. This includes, but is not limited to, designing, architecting, managing, developing, and testing their solution. In the designing and architecting phase, a developer is required to consult the members of their development team to ensure proper planning and compatibility. Each developer must have their functioning implementation thoroughly verified by another team member before merging their solution into the development branch. 

As a whole, the project development and testing is managed by Brady Shimp. Mr. Shimp manages the development of the project through the task board by creating, delegating, and monitoring the completion of tickets. 

\section{Budget}
% Describe the budget for the project including gifted equipment and salaries for 
% people on the project.

There is no budget actively established for the project. Pending official adoption of the product by the South Dakota School of Mines and Technology, proper withholdings will be made to support a year of operation of the product. All other profit will be allocated evenly among the student development team. 

The client has indicated possession of a \$40,000 grant budget for developing mixed reality course content and purchasing associated hardware. Client has purchased a variety of AR and VR hardware (such as the Microsoft HoloLens, Meta 2, and Oculus Rift) that is available for product testing through the student status of developers. Client's indicated intentions are to use a portion of said grant for purchasing the product's Early Adopters License. 

\section{Intellectual Property and Licensing}
% Describe the IP ownership and issues surrounding IP.

All intellectual property and ownership is retained by InTouch L.L.C.. 

Licenses for Visual Studio 2017 need not be purchased as students are already granted licensed access. All other third party resources utilized are available under the MIT License or likewise. 


\section{Sprint  Overview}
The development cycle that is being followed is a feature driven development.
The features and milestones were created at the beginning of the semester.
Sprint were handled in two week incriments and features were assigned to 
the sprints to be accomplished during the two weeks. The backlog consisted
of what was going to be accomplished in the coming weeks and what was not
completed in the last sprint. At the begining/end of each sprint, the 
developers would get together for a sprint planning/retrospective, to discuss
how to previous sprint went and how to handle the up coming one.


\section{Terminology and Acronyms}
% Provide a list of terms used in the document that warrant definition.  Consider 
% industry or domain specific terms and acronyms as well as system specific. 

\begin{itemize}
	\item Augmented Reality (AR): hardware and software that, together, superimpose computer-generated images on a user's view of the real world. Often, this composite view may be interacted with. 

	\item Virtual Reality (VR): hardware and software that, together, create a computer-generated simulation of a three-dimensional image or environment. Often, this simulation may be interacted with. 

	\item Mixed Reality (MR): the overlap in domain space of augmented reality and virtual reality. 

	\item Microsoft HoloLens: portable and cordless augmented reality viewing device. 

	\item Meta 2: augmented reality viewing device that must be connected to a computer and power outlet. 

	\item Mira Prism: augmented reality viewing device that leverages a user's mobile device.

	\item Oculus Rift: virtual reality viewing device. 

    \item QR Code: machine readable matrix barcode optical label.
    
    \item Unity: video game engine used in creating HoloLens apps.

	\item Cloud: off-site computing and digital storage resources accessed via the internet. 

    \item Azure: Microsoft Cloud hosting feature

	\item .fbx: model file type that may be rendered by most AR devices on the market. 
\end{itemize}


\section{Sprint Schedule}
As stated above, sprints were two weeks long. Each sprint was designed to accomplish a milestone. 
Below is the schedule of the sprints and what was accomplished during those sprints.

\begin{itemize}
	\item 9/18/2017   - Define milestones and user stories

	\item 10/2/2017   - Base website created with file upload ability
	
	\item 10/16/2017  - Wireframes, Presentation 1 documents, standalone file conversion
	
	\item 10/30/2017  - File upload/download, file conversion integrated
	
	\item 11/13/2017  - QR code generation, standalone advanced file conversion 
	
	\item 11/27/2017  - File conversion integration, create users, Presentation 2 documents
	
	\item 12/04/2017  - Finish documentation, second semester prep
	
	\item 12/13/2017  - MVP delivered to sponsor
	
	\item 04/27/2017  - Final product delivered to sponsor.
\end{itemize}

\section{Timeline}
Below is the timeline of accomplished milestones.

\begin{figure}[H]
\begin{center}
\includegraphics[width=1\textwidth]{./SprintGanattChart}
\end{center}
\caption{Sprint Timeline}
\end{figure}

\section{Development Environment}
%The basic purpose for this section is to give a developer all of the necessary 
%information to setup their development environment to run, test, and/or develop. 
Microsoft's Development environment was chosen for this project. This environment
was chosen for the following reasons:

\begin{itemize}
    \item Azure allows for simple cloud hosting and database integration.

    \item Student Licenses give certain paid feature of Azure for free.
    
    \item Access to Visual Studio is free through student accounts

    \item Developers were comfortable working with the ASP.NET MVC framework.

    \item Microsoft is the owner of the targeted device, The Hololens.

    \item Unity is used for HoloLens app development which has a C\# backing making it necessary to us Microsoft environment.

\end{itemize}

\section{Development IDE and Tools}
%Describe which IDE and provide links to installs and/or reference material. 
Many IDE and tools are used to create this project. Below is a list of those
tools with descriptions:

\begin{itemize}

    \item Website IDE and Tools
    \begin{itemize}
        \item Visual Studio 2017 Enterprise - Microsoft's code editor. Allows for easy Azure and database
        integration. Also allows for easy intallation of the Unity Development Tools.
        \item ASP.NET MVC - Microsoft's framework for creating websites.
        \item Azure Cloud Tools - Tools that allow for easy integration with Azure, able to install in Visual Studio Installer
    \end{itemize}

    \item File Conversion and Tools
    \begin{itemize}
        \item Autodesk's FBX SDK is required to export .fbx files.  It must be installed in a folder located in the project directory named \"FBX SDK\".  The download can be found at: 
        \url{http://usa.autodesk.com/adsk/servlet/pc/item?siteID=123112&id=26416244}.
        The Windows VS2015 version must be installed.

        \item Open Asset Import Library supports a wide variety of import and export file types.  The download can be found at: \url{http://assimp.org/main_downloads.html}.  Version 3.1.1 is what was used in the project.         
    \end{itemize}

    \item HoloLens Development Tools
    \begin{itemize}
        \item Unity Personal. Free version of Unity editor used for creating HoloLens app

        \item HoloToolkit - Free Set of tools you can download from an open source GitHub Project

    \end{itemize}
\end{itemize}


\section{Source  Control}
\subsection{Source Control Tool}
GitHub was used to provide source control and a private repository. This tool was chosen over others available as it is unanimously familiar to all developers. 

\subsection{Source Control Architecture}
The repository was created with the following three branches: Master, Develop and Release. 
Develop is used for work in progress. After features have been verified through thorough testing, they will be merged into the Release branch for beta usage. Once verified in Release, features are merged into the Master branch is reserved for the final deliverable. 


\subsection{Source Control Etiquette}
The following steps represent the code flow process: 

\begin{enumerate}
\item Pull latest code from the develop branch
\item Implement feature
\item Push to Develop and assign other developer for QA
\item Once properly tested, merge feature into Release branch
\item Once feature verified in Release, merge feature into Master
\end{enumerate}

\subsection{Developer Contributions}
Figure \ref{Contributions} shows a snapshot of repository activity and developer contributions:

\begin{figure}[H]
	\centering
	\includegraphics[width=\textwidth]{Contributions.png}
	\caption{Developer Contributions} 
	\label{Contributions}	
\end{figure}



\section{Dependencies}
%Describe all dependencies associated with developing the system. 
\paragraph{Website}
\begin{itemize}
    \item Visual Studio 2017 Enterprise - Microsoft's code editor. Allows for easy Azure and database
    integration. Also allows for easy intallation of the Unity Development Tools.
    \item ASP.NET MVC - Microsoft's framework for creating websites.
    \item Azure Cloud Tools - Tools that allow for easy integration with Azure, able to install in Visual Studio Installer
\end{itemize}

\paragraph{File Conversion}
\begin{itemize}
    \item Autodesk's FBX SDK is required to export .fbx files.  It must be installed in a folder located in the project directory named \"FBX SDK\".  The download can be found at: 
    \url{http://usa.autodesk.com/adsk/servlet/pc/item?siteID=123112&id=26416244}.
    The Windows VS2015 version must be installed.

    \item Open Asset Import Library supports a wide variety of import and export file types.  The download can be found at: \url{http://assimp.org/main_downloads.html}.  Version 3.1.1 is what was used in the project.         
\end{itemize}

\paragraph{HoloLens App}
\begin{itemize}
    \item Unity Personal. Free version of Unity editor used for creating HoloLens app

    \item HoloToolkit - Free Set of tools you can download from an open source GitHub Project
\end{itemize}

\section{Build  Environment}
Azure is used to build and deploy the website. Once the tools are downloaded and an Azure account is created, 
all that is needed is to right-click on the ASP.NET MVC project and select publish. The website is then hosted on Azure and
can be accessed by all.
\section{Development Machine Setup}

Pull the git repository with both the Website and File Conversion code located at: \url{https://github.com/SavoySchuler/ARFE}

% If warranted, provide a list of steps and details associated with setting up a 
% machine for use by a developer. 

The following are the requirement for a development machine:
\begin{itemize}
    \item Windows 10
    \item Visual Studio 2017 Enterprise with ASP.NET MVC, Azure and Unity tools
    \item Unity Personal
    \item HoloToolkit-Unity Set of tools for Hololens development with unity. Free for use. Download from the GitHub page.
\end{itemize}

\paragraph{File Conversion}

There are two libraries that need to be installed: Assimp and FBX SDK.

\subparagraph{FBX SDK}

\begin{enumerate}
    \item Go to the following link and install the FBX SDK.
    \begin{itemize}
        \item \url{http://usa.autodesk.com/adsk/servlet/pc/item?siteID=123112&id=26416244}
        \item Use the installer under: Windows / FBX SDK 2018.0 VS2015
    \end{itemize}

    \item Copy the FBX SDK folder to the directory with the source code (the deepest FileConverion folder)
    \begin{itemize}
        \item At the location of the installiation, the file structure should be: Autodesk/FBX/FBX SDK/
        \item Copy the FBX SDK folder to the source code directory.
    \end{itemize}
\end{enumerate}

\subparagraph{Assimp}

\begin{enumerate}
    \item Go to the following link and install the Assimp software.
    \begin{itemize}
        \item \url{http://assimp.org/main_downloads.html}
        \item Version 3.1.1 is what was used during development.
    \end{itemize}

    \item From the installiation location, copy the \"Assimp\" folder into the deeper FileConversion directory.
    \begin{itemize}
        \item Paste the folder into the same folder as the source code of the File Conversion.
    \end{itemize}

    \item Copy FileConverion/Assimp/bin/x86/assimp-vc140-mt.dll to the same directory as the source code.
\end{enumerate}