\section{HoloLens}

    \subsection{Research & Tutorials}

    At the end of the first semester and beginning of the second, the development team pursued the idea of creating a custom HoloLens application.
    Starting research into this proved difficult, as the all of this technology was just release and poorly documented. The following
    Following links are the tutorials that were used to learn HoloLens Development.

    \begin{itemize}
        \item Environment Setup - https://docs.microsoft.com/en-us/windows/mixed-reality/install-the-tools
        \item Unity Basic Tutorial - https://unity3d.com/learn/tutorials
        \item Basic HoloLens App Tutorial - https://hololens.reality.news/how-to/hololens-dev-101-build-basic-hololens-app-minutes-0175021/
        \item Advanced HoloLens App Tutorial - https://next.reality.news/how-to/hololens-dev-101-building-dynamic-user-interface-part-1-setup-0179747/
    \end{itemize}

    \subsection{Unity}
    Using the tutorials mentioned above, the development team learned the basics of the Unity. Unity is a game engine with a C\# backing.
    Adding an object in Unity, creates a skeleton object in C\# that is tied to the object. Generic event handlers can be assigned to the object through
    the editor, or custom event handlers could be created by manipulating the code using Visual Studio. 

    \subsection{HoloToolKit}
    HoloToolKit is a Unity/C\# library that duplicated many of the objects that come with basic Unity, such as, viewer camera or interactable object.
    With this toolkit, a simple application with basic gaze interaction was created.
    
    \subsection{Development Problems}
    The development team started to face difficulties based on the tools that we were using
    The main walls that were hit during development were Unity's handling of .fbx file, HoloToolKit's room mapping, and overall performance of the app.
    The following is a description of why those were problems
    \begin{itemize}
        \item Unity's handling of .fbx files: Unity requires that 3D models be loaded before compilation of the code, because unity converts the models to their own filetype.
        This is a large problem because for the app to provide the correct functionality it needs to load and display .fbx files at run time.
        \item HoloToolKit's room mapping: It only uses the Hololens room mapping software to keep track of the Hololens position.
        This mean the models would clip through wall and floors. This was nessessary for the application, but it is a functionality provided by Microsoft default viewing software.
        \item Performance: The viewer with the HoloToolKit showed lag. This caused the models to shift rapidly as you moved around them. 
    \end{itemize}

    The custom HoloLens app was dropped due to these reasons. It became clear that an app with the basic functionality required for the project was a long way out.
    It made more since to make sure the website worked with the Microsoft tool chain, because it offered the functionality needed immedatly.
    This also allows the users to use the latest and greatest viewing software without having to dedicate a team to continuous updating of an application.



    

        