% Research / Hardware / ARDevices

\subsubsection{AR Devices}

	Augmented Reality devices are different than VR Devices in that the applications are presented on top of what the user would normally see, not blanking out the screen and recreating the world.  Currently, there are two main headset contenders in the AR space, Microsoft's HoloLens and the Meta 2.  Mobile devices are now becoming more popular for AR.

	\paragraph{HoloLens}

		The HoloLens is a wireless AR headset developed by Microsoft.  The developer edition of the HoloLens was released in March 2016, and has not had any further releases since.  They (Microsoft) are currently developing the HoloLens 2, the next iteration of the hardware.  The original version has some user complaints such as a small viewing port, that will be addressed in version two.  

	\paragraph{Meta 2}

		The Meta 2 is another headset.  It is wired, so a computer on par with the requirements for the Oculus VR headset is required.  Being wired is a downside for mobility, in that the user must always be close to a computer.  However, it is also a good thing in that the headset can use the more powerful processing on the computer rather than having the chips on board the headset.

	\paragraph{Mobile Phone}

		AR has recently been moving to the mobile phone space.  Both iOS and Android have AR offerings, ARKit and ARCore respectively.  ARKit was released by Apple in December 2017, and ARCore was released by Google in February 2018.  Since the offerings are so new, the number of applications developed with AR is still limited.  
		
		Mobile devices potentially have a larger consumer base than the dedicated headsets.  Most people have a smart phone (almost all running iOS or Android), and if AR was enabled on those devices, a large portion of the population would have AR capabilities.  Also, the mobile devices are more portable than the dedicated headsets.  A mobile phone can be carried in a pocket or small bag while the headsets must either be worn or be carried in a specialized carrying case.  
		
		One main drawback on running AR through mobile phones is the limited computational power.  A dedicated headset can have specialized hardware (including increased graphics processing) as opposed to a phone.  Also, all of the software on the headsets is designed for AR use.  So the user interfaces are designed with AR in mind, which may or may not be the case for the mobile devices.