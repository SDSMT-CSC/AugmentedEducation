% Research / Hardware / VRAR / VRDevices

\subsubsection{VR Devices}

	Currently (as of May 2018) there are two leading VR headsets: the Oculus Rift and HTC Vive.  The HTC Vive comes in two main forms, the regular Vive and the Vive Pro.  The Vive Pro is a wireless version of the Vive that has better specifications than the regular version.  The Vive sells for around \$500 and the Vive pro for \$800. The Oculus Rift (or Oculus for short) is produced by Facebook and sells for around \$400.

	\paragraph{HTC Vive}

		The HTC Vive is produced by HTC (headquartered in Taiwan).  It is considered the higher tier device (over the Oculus) but also has a higher price tag.  It heavily relies on the SteamVR software produced by Valve.  Tracking for the Vive is done using two "Lighthouses" placed in opposite corners of the room.  They work with the headset and controllers to determine where and what orientation the devices are in.  The information is then sent to the computer and displayed in the application.  The Vive software is heavily integrated with the SteamVR software.  SteamVR is Valve's integration with their popular gaming platform Steam.  It allows users to access their Steam account from within the VR setting to access the store, message friends, and play games.  The connection to the Vive takes place from the headset to a provided "Link box", and from the link box to the computer.  Overall the ports required on the computer are: 1 USB and 1 HDMI.

	\paragraph{Oculus Rift}

		As stated before, the Oculus is produced by Facebook (headquartered in California).  It is a cheaper option compared to the Vive.  It uses its own software to run the headset, but can use SteamVR.  Tracking is performed using two sensors that are placed on either side of the user's computer.  They each take up a USB 2.0 port. Overall, the setup takes 3 USB ports (2x 2.0, 1x 3.0) and 1 HDMI port.
