% Research / Hardware / PC

\subsection{PC}

Since the school purchased an Oculus and Meta 2, there was a need for a school owned computer that could run them both.  Also, Brent Deschamp was requesting a computer that can run the hardware in order to begin development of content for the website that could be used in his Calculus 3 course.  The most important factor in running the specialized hardware is the Graphics Processing Unit (GPU).  The recommended GPU requirement from Oculus was an NVidia GTX 1060 or above.  

The original intent at the beginning of the project was to purchase NVidia GTX 1080's (which are better than 1060's).  However, after December of 2017, the price of GPU's increased a lot (in some cases greater than a factor of two).  Therefore, at the time of purchasing the hardware, the price range for the acceptable cards was:

\begin{itemize}
    \item GTX 1080: \$700 - \$1300
    \item GTX 1070: \$550 - \$700
    \item GTX 1060: \$300 - \$500 
\end{itemize}

In order to save the grant money, the original intent of purchasing a 1080 to run the hardware was not pursued.  Instead, an upper tier 1060 was purchased.  The full specifications of the card are listed below:

\begin{itemize}
    \item ASUS Dual - GTX 1060
    \item 6 GB DDR5 RAM
    \item 2x HDMI Ports
    \item 2x Display Ports
    \item 1x DVI Port
    \item PCIe 3.0 x 16
\end{itemize}

In addition to the GPU, a Solid State Drive (SSD) was purchased.  SSD's are like spinning disk hard drives except use flash memory instead of magnetic disks.  They are faster their counterparts, so it decreases boot time and allow for better access during program execution.  The computer that was given to the group to outfit with the GPU only had a spinning disk hard drive, so an SSD was purchased for quality of life improvements while running the hardware.

Overall the full system specifications for the computer are:

\begin{itemize}
    \item CPU: Intel i7 ???????
    \item GPU: GTX 1060, 6GB
    \item Storage: 500 GB SSD, 500 GB Hard Disk
    \item RAM: 16 GB
\end{itemize}

In order to run the hardware, and develop content for the project, software was needed.  The Oculus and Meta both required software from the hardware manufacturers.  The full list of software initially installed is below:

\begin{itemize}
    \item Oculus
    \begin{itemize}
        \item Runs the Oculus Rift VR headset.
    \end{itemize}
    \item Meta 2 SDK
    \begin{itemize}
        \item Used to run the Meta 2 AR headset.
        \item Unable to get the software to work due to an error that was occurring during setup.  Error described to Brent on delivery of the computer.
    \end{itemize}
    \item Blender
    \begin{itemize}
        \item Used to visualize some models, including the DAE file type.
    \end{itemize}
    \item Notepad++
    \begin{itemize}
        \item Used to visualize the files on the text level, useful to view/modify .obj files.
    \end{itemize}
    \item Maple
    \begin{itemize}
        \item Used to generate 3D plots of mathematical functions.
    \end{itemize}
\end{itemize}