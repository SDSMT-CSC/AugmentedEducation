
\section{Requirements and Design Constraints}

% we could include an intro here

\subsection{System Requirements}

% What are they? How will they impact the potential design? Are there alternatives
The basic system requirements to use the website are to have a web browser 
installed with internet access.  The user must have access to modeling software 
or a method to create/provide 3D models to the website.

In order to fully use the product, a user must have an Augmented Reality device.
Each device may have different system requirements. Some AR devices such as the 
HoloLens or AR compatible smartphones will not require a computer to run off. 
However, some AR devices require to be tethered to a computer. Usually, 
manufacturers detail a set of quite high minimum specifications for a computer 
that can be used

The Meta 2 is an example of a headset that requires a separate computer in order
to run.  In our research we found that the Meta 2 specified one of the highest 
recommended specifications among popular VR headsets. This makes it a good 
example for system specifications that would run an AR application smoothly. 
The minimum and recommended specifications are listed below. The list was crated
in November 2017, and can be found in table 
\ref{table:metatwosystemrequirements}.

\begin{table}[H]
	\centering
	\begin{tabular}{ | c | c | c | }
		\hline
		& Minimum & Recommended \\ \hline
		OS & Windows 10 (64 bit) & 	Windows 10 (64 bit) \\ \hline
		CPU & Intel i7-4770 & Intel i7-6700 \\ \hline
		RAM & 8GB DDR3 & 16GB DDR4 \\ \hline
		GPU & NVIDIA GTX 960 & NVIDIA GTX 970 \\ \hline
		Hard Drive & 2GB Free Space & 2GB+ Free Space \\ \hline
		I/O Ports & 1X HDMI 1.4b and 2X USB 3.0 ports & 1X HDMI 1.4b and 2X USB 3.0 ports \\ \hline
		3D Engine & Unity 5.6 or higher & Unity 5.6 or higher \\ \hline
	\end{tabular}

	\caption{Meta 2 System Requirements}
	\label{table:metatwosystemrequirements}
\end{table}

More up to date requirements can be found on the Meta 2 website at: 
\url{https://buy.metavision.com/}

\subsection{Network Requirements}

Being able to take advantage of the features in this product will need a network
that is able to easily upload and download 3D models. The user will need to have
both their computer and AR viewer configured for network access.

\subsection{Development Environment Requirements}
% What are they? Is the system supposed to be cross-platform?
\begin{itemize}
	\item Windows 10 - To be able to develop for the HoloLens.
	\item Visual Studio 2017 Enterprise with ASP.NET MVS, Azure and Unity tools.
	\item Unity Personal
	\item HoloToolkit - Unity Set of tools for HoloLens development with Unity.
	\item Assimp library
	\item FBX SDK
	\item Android Studio
\end{itemize}


\subsection{Project Management Methodology}
% The stakeholders might restrict how the project implementation will be managed.
%  There may be constraints on when design meetings will take place. There might
% be restrictions on how often progress reports need to be provided and to whom.

The Senior Design I team structure for this project was self-organized, having decided on the Agile development methodology and decided on Brady Shimp as a Scrum Master and Cheldon Coughlen as the Team Lead. The team had weekly status report meetings with our client representatives and faculty advisors, Dr. McGough and Dr. Karlsson. At these meetings the team provided progress reports and discussed any design modifications or complications moving forward.\\

The Senior Design II team structure for this project was refined from experiences, both good and bad, taken from the Senior Design I structure.  The team agreed that the Agile development methodology was too rigid and ineffective for a team with as many members and as many different times of availability as there were.  The team instead opted for a Feature Driven Development (FDD) methodology, that employed certain aspects of the Agile system that were deemed effective such as listing expected completion times for feature tasks and tracking productivity via the Scrum project board method. This switch allowed the team to diversify into two sub-teams, one for mobile application development and one for continued web development, that each maintained structure and accountability within themselves.  The decided leaders of the mobile application team and the continued web development team were Kenneth Petry and Brady Shimp, respectively.\\

Consistencies across both courses of Senior Design I and Senior Design II were the Scrum style issue tracking, GitHub for the bulk project management tools, Google Team Drive for collaborative shared materials and presentation preparation, team communication through the Discord chat application, weekly status meetings with the project clients, and a structured task delegation system with a single point of authority.

% \section{Specifications}


% Any specifications that need to be understood? Put it here.


\section{Product Backlog}


This is the entire backlog for the project, including each actionable item we 
have completed or will be working on. New items and checkpoints will be added 
here as they come up.

\begin{itemize}
\item Take in files from Maple and convert them to stored file type.
\item Create website to upload and download 3D models
\item \textbf{Checkpoint One:} Create wire frames, documentation for presentation 1.
\item Host conversion software on the website.
\item Test viewing 3D objects on the HoloLens.
\item Manually create unique QR codes for each file. 
\item \textbf{Checkpoint Two:} Create wire frames, demos, documentation for presentation 2.
\item Automatic QR code generation.
\item Detect QR code and download file.
\item Create and manage user profiles.
\item Allow users to control visibility of uploaded models.
\item HoloLens application development to view downloaded 3D model and interact 
with it.
\item \textbf{Final Checkpoint:} Prepare materials for design fair.
\end{itemize}

\subsection{Backlog Tracker}

We manage our backlog using GitHub's project board. Using GitHub's system is 
convenient as we also use it to host our code repository. Items from our backlog 
are added to the backlog column of the project board. They are sometimes broken 
up into smaller issues if the feature is large and can be worked on by multiple 
developers.

When the issue is ready to be worked on, it is taken from the backlog and 
assigned to a developer. The project board has three other sections: 
'Work in Progress', 'QA', and 'Ready' as seen in Figure 
~\ref{fig:ProjectTaskBoard}. Once the developer begins working on it, 
they move it to 'Work in Progress'. When they have the issue taken care of, they
move it to 'QA' and assign it to a developer that has not worked on the issue, 
but has the background to test it. Once it is verified that the issue has been 
fixed, it is moved to 'Ready'. If any problems arise while testing the solution,
it is communicated to the developers who worked on it and the issue is moved 
back to 'QA'.

\begin{figure}[H]
    \centering
    \includegraphics[width=\textwidth]{ProjectTaskBoard.png}
    \caption{Project Task Board}
    \label{fig:ProjectTaskBoard}
\end{figure}

By default, just the development team has access to the Sprint and Product 
Backlog on GitHub, but through stand-alone documentation, presentations, and 
other communication, we share this information with our other stakeholders. 
We can provide access to the actual backlog we maintain on the GitHub repository
on request.

\subsection{Sprints}
The project is encompassed by 13 Sprints - 6 in the Fall semester and 7 in the 
Spring semester. Each sprint is defined to be two weeks long. 
For each sprint, we define a backlog and assign developers to tasks. We also 
define deliverables that will be prepared by the end of the sprint. At the end 
of each sprint, we reflect on the successes and failures, and look forward to 
what needs to happen in subsequent sprints.


\section{Proof of Concept Results}


To some extent, the work done in our first semester is centered on giving our 
stakeholders a proof of concept so that they can begin to drive the requirements
of the project in the future. We had a couple of meetings with faculty to demo 
what we were able to create in VR and AR so that they could see the capabilities
of the technology, and think about how they could integrate it in their 
classroom. At the end of our first semester of development, we will have an MVP
to demonstrate the proof of concept. Our presentations through the semester have 
contained wire frames and demos that reflect the features our MVP will have.

With regards to research and system design, the team has been doing both
concurrently. This is possible since Dr. McGough provided the team with a 
high-level view of the entire system and the data flow before we began the 
project. This meant that the team able to focus on researching and implementing 
smaller pieces of the system. 


\section{Supporting Material}


As mentioned earlier, the original mobile computing grant proposed by Dr. 
McGough and other SD Mines faculty is the main source that has driven the
specifications of the product. We have provided the grant in 
~\autoref{ch:support} so that the reader has a good understanding of the
motivation behind this product, and what the long-term goals are.

