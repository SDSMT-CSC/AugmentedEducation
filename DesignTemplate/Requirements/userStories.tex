\section{User Stories}

User stories are collected through conversations with the stakeholders and
regularly displaying what the team has accomplished. The senior design team had
weekly meetings with the client to review progress on the project. The team met
with other faculty related to the project as necessary in order to receive
feedback.

The starting point for the requirements and user stories for this project came
from the a grant proposal by Dr. McGough and other SD Mines faculty,
which can be found in ~\autoref{ch:support}. Over the course of several
meetings and conversations with the professors on the grant, the
requirements for the product were defined.

An example user story from the grant is as follows: "As a Calc III
professor, I would like to provide visualizations of 3D objects so that
students can find the volume of an object by being able to view the object
from different angles, and being able to slice the object's volume."

Some base requirements identified for the platform include:
\begin{itemize}
	\item A website to manage files
	\item Support for 3D files exported from CAD programs commonly used in STEM
	\item Automatic file conversion
	\item Render files on the Microsoft HoloLens
	\item Render files on an Android mobile device
\end{itemize}

At the advice of faculty advisors, the user stories were grouped into three 
development rounds. User story defines a test as appropriate. More detail on tests can be found in Chapter \ref{ch:testing}.

\subsection{Phase 1: Senior Design I}

User Stories:
\begin{itemize}
	\item As an SD Mines faculty member, I want to:
		\begin{itemize}
			\item Upload a 3D file to the cloud.
			\item Retrieve the file in the proper format.
			\item Download a QR code for a model.
			\item Log in to a secure account and view my uploaded files.
			\item Choose files to be private or public.
			\item Browse and download public files.
		\end{itemize}
\end{itemize}
Tests:
\begin{itemize}
	\item Conversion of multiple types of 3D models.
	\item Viewing and manipulating the models on the HoloLens.
	\item Generation of a unique QR code associated with each model.
	\item Upload and download process for a 3D model.
\end{itemize}

\subsection{Phase 1: Senior Design II}

User Stories:
\begin{itemize}
	\item As a user, I want to:
	\begin{itemize}
		\item View surface materials on 3D model.
		\item Use an Android phone for scanning QR codes and viewing models.
		\item Switch between models quickly in the AR device.
	\end{itemize}
\end{itemize}
Tests:
\begin{itemize}
	\item Upload a file and verify that the AR tag received links to the correct
	model.
	\item Test privacy rules and visibility for files by creating multiple 
	profiles with different privacy settings.
	\item Gather 3D models of multiple file types and test the rendering of 
	textures and surface materials.
	\item Verify that privacy settings work and users are able to download files
	according to permissions on those files.
\end{itemize}