\section{User Stories}


User stories are collected through conversations with the stakeholders and
regularly displaying what we have accomplished. Our senior design team has
weekly meetings with our client to review progress on the project. We meet
with other faculty related to the project as necessary in order to receive
specific feedback.

The starting point for the requirements and user stories for this project comes
from the original grant proposal by Dr. McGough and other SD Mines faculty
which can be found in ~\autoref{ch:support}. Over the course of several
meetings and conversations with the professors on the grant, we refined the
requirements for our product.

An example user story from the grant, which is a close representation of the
use case we are trying to support with our MVP, is as follows: "As a Calc III
professor, I would like to provide visualizations of 3D objects so that
students can find the volume of an object by being able to view the object
from different angles, and being able to slice the object's volume"

Some base requirements we have identified for our platform include:
\begin{itemize}
	\item Must have a website to manage files
	\item Must support 3D files exported from CAD programs commonly used in STEM
	\item Must automatically convert files to AR device compatible format
	\item Must render files on the Microsoft HoloLens
	\item Must render files on an Android mobile device
\end{itemize}

At the advice of our faculty advisor, we grouped our user stories into three 
development rounds. For each user story we discuss the testability of the 
feature, and include the test if it is. More detail on tests can be found in the testing section, Chapter \ref{ch:testing}.

% This section can really be seen as the guts of the document.  This section
% should be the result of discussions with the stakeholders with regard to the
% actual functional requirements of the software.  It is the user stories that
% will be used in the work breakdown structure to build tasks to fill the
% product backlog for implementation through the sprints.

% This section should contain sub-sections to define and potentially provide a
% breakdown of larger user stories into smaller user stories. Each component
% must have a test identified, meaning you need to know how you plan to test 
% it. If a requirement is not testable, then some justification needs to be 
% made on why the requirement has been included. The results of the tests 
% should go in the testing chapter.

\subsection{Phase 1: Senior Design I}

User Stories:
\begin{itemize}
	\item As an SD Mines faculty member, I want to:
		\begin{itemize}
			\item Upload a 3D file to the cloud.
			\item Retrieve the file in my needed format.
			\item Download a QR code for my model.
			\item Log in to my secure account and view my uploaded files.
			\item Choose my files to be private or public.
			\item To browse and download files that are public.
		\end{itemize}
\end{itemize}
Tests:
\begin{itemize}
	\item Test conversion of multiple types of 3D models.
	\item Test viewing and manipulating the models on the HoloLens.
	\item Test generation of a unique QR code associate with each model.
	\item Test the upload and download process for a 3D model.
\end{itemize}

\subsection{Phase 1: Senior Design II}

User Stories:
\begin{itemize}
	\item As a user, I want to:
	\begin{itemize}
		\item View surface materials.
		\item Use an Android phone for scanning QR codes and viewing models.
		\item Use an Android phone for scanning QR codes and viewing models.
		\item Switch between models quickly in the AR device.
	\end{itemize}
\end{itemize}
Tests:
\begin{itemize}
	\item Upload a file and verify that the AR tag received links to the correct
	model.
	\item Testing the security of an account may be out of the scope of this 
	phase. We will strive to research and maintain best practices while 
	implementing features.
	\item Test privacy rules and visibility for files by creating multiple 
	profiles with different privacy settings.
	\item Gather 3D models of multiple file types and test the rendering of 
	textures and surface materials.
	\item Verify that privacy settings work and users are able to download files
	according to permissions on those files.
\end{itemize}