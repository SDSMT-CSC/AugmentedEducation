\section{User Stories}


User stories are collected through conversations with the stakeholders and
regularly displaying what we have accomplished. Our senior design team has
weekly meetings with our client to review progress on the project. We meet
with other faculty related to the project as necessary in order to receive
specific feedback.

The starting point for the requirements and user stories for this project comes
from the original grant proposal by Dr. McGough and other SD Mines faculty
which can be found in ~\autoref{ch:support}. Over the course of several
meetings and conversations with the professors on the grant, we refined the
requirements for our product.

An example user story from the grant, which is a close representation of the
use case we are trying to support with our MVP, is as follows: "As a Calc III
professor, I would like to provide visualizations of 3D objects so that
students can find the volume of an object by being able to view the object
from different angles, and being able to slice the object's volume"

Some base requirements we have identified for our platform include:
\begin{itemize}
	\item Must have a website to manage files
	\item Must support Maplesoft generated 3D files
	\item Must automatically convert files to AR device compatible format
	\item Must render files on the Microsoft HoloLens
\end{itemize}

At the advice of our faculty advisor, we grouped our user stories into three 
development rounds. For each user story we discuss the testability of the 
feature, and include the test if it is. More detail on tests can be found in the testing section, chapter \ref{ch:testing}.

% This section can really be seen as the guts of the document.  This section
% should be the result of discussions with the stakeholders with regard to the
% actual functional requirements of the software.  It is the user stories that
% will be used in the work breakdown structure to build tasks to fill the
% product backlog for implementation through the sprints.

% This section should contain sub-sections to define and potentially provide a
% breakdown of larger user stories into smaller user stories. Each component
% must have a test identified, meaning you need to know how you plan to test 
% it. If a requirement is not testable, then some justification needs to be 
% made on why the requirement has been included. The results of the tests 
% should go in the testing chapter.

\subsection{Round One}

User Stories:
\begin{itemize}
	\item As an SD Mines faculty member, I want:
		\begin{itemize}
			\item To upload a Maple 3D file to a cloud server.
			\item Have the Maple 3D file to be automatically converted to an AR 
			Tag.
			\item Download the AR Tag for my model from the cloud server.
		\end{itemize}
\end{itemize}
Tests:
\begin{itemize}
	\item Test conversion of multiple types of 3D models.
	\item Test viewing and manipulating the models on the HoloLens.
	\item Test generation of a unique AR tag for each model.
	\item Test the upload and download process for a 3D model.
\end{itemize}

\subsection{Round Two}

User Stories:
\begin{itemize}
	\item As a faculty member, I want:
	\begin{itemize}
		\item A Maple file to be automatically converted into an AR Tag on a 
		cloud server.
		\item To log in to my secure account on the cloud server and view my 
		uploaded files.
		\item To choose my files to be private, shared with some, or public.
	\end{itemize}
	\item As a user, I want:
	\begin{itemize}
		\item To be able to view surface materials.
		\item To be able to detect an AR tag through a Microsoft HoloLens and 
		have the 3D model rendered.
	\end{itemize}
\end{itemize}
Tests:
\begin{itemize}
	\item Upload a file and verify that the AR tag received links to the correct
	model.
	\item Testing the security of an account may be out of the scope of this 
	project. We will strive to research and maintain best practices while 
	implementing features.
	\item Test privacy rules and visibility for files by creating multiple 
	profiles with different privacy settings.
	\item Gather 3D models of multiple file types and test the rendering of 
	textures and surface materials.
\end{itemize}

\subsection{Round Three}

User Stories:
\begin{itemize}
	\item As a faculty member, I want:
	\begin{itemize}
		\item To browse and download files that are public or shared with me.
		\item To be able to build models for a ‘time lapse layered’ 
		presentation. Swipe right for more detail, left for less
	\end{itemize}
	\item As a user, I want: 
	\begin{itemize}
		\item To be able to slice a 3D model and view a section of the model.
		\item To be able to be able to switch between models quickly in the AR 
		device.
	\end{itemize}
\end{itemize}
Tests:
\begin{itemize}
	\item Verify that privacy settings work and users are able to download files
	according to permissions on those files.
	\item Test gesture control in the HoloLens application.
	\item Use multiple 3D models and multiple examples of slicing to create 
	sections.
\end{itemize}