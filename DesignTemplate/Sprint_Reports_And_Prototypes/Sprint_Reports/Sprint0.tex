% !TEX root = ../../DesignDocument.tex

\section{Sprint 0 Report}
\label{sec:Sprint0_report}
    \subsection{Sprint Backlog}
    \label{sec:Sprint0_backlog}
        \begin{itemize}
            \item Define the toolchain
            \item Define the Minimum Viable Product (MVP)
            \item Create a data flow diagram to detail the MVP            
            \item Identify use cases and create user stories
            \item Estimate a timeline for the development process
        \end{itemize}

    \subsection{Successes}
    \label{sec:Sprint0_successes}
        \begin{itemize} \item Defined the MVP \end{itemize}
        \hspace{7mm}
        Upload a Maplesoft 3D object file to a website.  Have the website perform an automatic file conversion and 
        present the user with an AR Tag.  When the user views the AR Tag through an AR device, download and render 
        the 3D object file.

        \begin{itemize} \item Decided to develop with the Microsoft Hololens being the primary supported AR device. \end{itemize}
        \hspace{7mm} 
        A conscious decision was made in accordance with current technologies and the Mobile Computing Grant awarded 
        to the South Dakota School of Mines and Technology that this service aims to satisfy to keep development 
        centered around the Hololens and other Microsoft supported or compatible tools and services.

        \begin{itemize} \item Decided to use Azure for cloud hosting services \end {itemize}
        \hspace{7mm}The decision was between using Azure cloud services or Amazon Web Services.
        Azure was voted as the better of the two options on the grounds that the primary device we intend to make work
        using this service is the Microsoft Hololens and compatibility conflicts should be avoided.

        \begin{itemize} 
            \item Created initial user stories 
            \begin{itemize}
                \item As a faculty member, I want:
                \begin{itemize} 
                    \item to upload a maple file to a cloud server.
                    \item the maple file to be automatically converted to an AR tag on the cloud server.
                    \item to be able to download the AR tag for my document from the cloud server.
                \end{itemize}
                \item As a user of this product, I want: 
                \begin{itemize}
                    \item to be able to view the AR tag through a Microsoft Hololens to render a 3D model.
                \end{itemize}
            \end{itemize}
            \item Estimated tentative development timeline.
        \end{itemize}

    \subsection{Failures}
    \label{sec:Sprint0_failures}
        \begin{itemize} \item None \end{itemize}

    \subsection{Deliverables}
    \label{sec:Sprint0_deliverables}
        \begin{itemize} 
            \item Definition of the MVP
            \item Data flow diagram            
            \item User stories
            \item Defined toolchain
            \item A production timeline (tentative)
        \end{itemize}

    \subsection{Sprint Review}
    \label{sec:Sprint0_review}
        \hspace{7mm}
        All items on the sprint backlog were resolved in a timely, effective manner.

    \subsection{Sprint Retrospective}
    \label{sec:Sprint0_retrospective}
        \hspace{7mm}
        As this is the `setup' sprint, tasks and discussions were primarily limited to topics of planning and preparation.
        The team has been split into two subteams, a web team and a file conversion team.\\ 

        The web team will be responsible for the development and management of the ASP.NET website.  This includes the
        structure and design of the website,  as well as the content, account security, organization, and data 
        storage/access.  Members are Daniel Hodgin, Savoy Schuler, and Brady Shimp.\\

        The file conversion team will be responsible for research into file type and structure of commonly used 3D modelling
        softwares such as Maplesoft (TM), and file types that are render-able in commonly used Augmented Reality viewing devices.
        The expected result of the research is to devise a system to convert common 3D modelling software file types into either
        a common file type that is made render-able by commonly used Augmented Reality viewing devices, or a ubiquitous format
        that can then be exported to the necessary file format, depending on which Augmented Reality viewing device is 
        requesting the file. Members are Aaron Alphonsus, Cheldon Coughlen, and Kenneth Petry.