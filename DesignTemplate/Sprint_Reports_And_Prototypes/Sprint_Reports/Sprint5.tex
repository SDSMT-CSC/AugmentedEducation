% !TEX root = ../../DesignDocument.tex

\section{Sprint 5 Report}
\label{sec:Sprint5_report}
    \subsection{Sprint Backlog}
    \label{sec:Sprint5_backlog}
        \begin{itemize}
            \item Reserve a development workspace
            \item Prepare and give the second client presentation            
            \item Research into COLLADA file type for Maple textures
            \item Continue work to implement automatic QR code generation
            \item Update project documentation in accordance to the Senior Design Fall 2017 requirements
        \end{itemize}

    \subsection{Successes}
    \label{sec:Sprint5_successes}
        \begin{itemize}
            \item COLLADA file type research
            \item The second client presentation
            \item Update project documentation for Senior Design
        \end{itemize}

    \subsection{Failures}
    \label{sec:Sprint5_failures}
        \begin{itemize}
            \item Unable to reserve a development workspace
            \item Continue to implement automatic QR code generation
        \end{itemize}

    \subsection{Deliverables}
    \label{sec:Sprint5_deliverables}
        \begin{itemize}
            \item The second client presentation
            \item Project documentation for the 2017 Fall semester
        \end{itemize}

    \subsection{Sprint Review}
    \label{sec:Sprint5_review}
        \hspace{7mm}
        A private development workspace is still yet to be reserved.\\

        Unique QR codes are able to be manually generated given a file path.  A plan exists for automating
        the process but is not yet implemented.\\
        
        COLLADA file type research shows that the COLLADA DAE file format should be the intermediary file type.

    \subsection{Sprint Retrospective}
    \label{sec:Sprint5_retrospective}
        \hspace{7mm}
        The development team is in agreement that this was a successful sprint.  The amount of
        content developed during this sprint was minimal in lieu of items such as the client 
        presentation and ensuring the completeness of the documentation required for the 2017
        Fall semester.\\
        
        On the subject of finding a private development workspace, the decision remains unanimous that until one 
        is able to be reserved, working sessions involving the AR and VR equipment that is hardware dependent 
        will have to be scheduled with a team member who owns a laptop that meets the minimum hardware
        specifications.\\

        On the subject of fully implementing QR code generation, the decision remains that 
        the feature is not crucial to the function of the MVP and can be postponed until after the delivery
        of the MVP to be implemented.\\

        On the subject of COLLADA DAE file type research, The upside of using COLLADA is that it doesn't just 
        contain `geometry data' but also information about motion, light sources, textures etc. COLLADA also 
        has a lot of applications that support it - meaning that being able to export to it and convert it to 
        other formats will probably be easy. The downside to using COLLADA will probably not be an issue. 
        COLLADA has a complex file structure which makes parsing and loading it a challenge. However, at least 
        at this point, it doesn't look like this project will need that. Also, there are a few libraries to 
        help out with this, if needed.\\