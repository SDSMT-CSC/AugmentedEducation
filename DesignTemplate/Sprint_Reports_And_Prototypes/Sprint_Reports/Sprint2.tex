% !TEX root = ../../DesignDocument.tex

\section{Sprint 2 Report}
\label{sec:Sprint2_report}
    \subsection{Sprint Backlog}
    \label{sec:Sprint2_backlog}
    \begin{itemize}
        \item Prepare for the first client presentation
        \item Working file conversion in a standalone application
        \item Test that simple file upload to the ASP.NET web root is operable
        \item Define the minimum specifications for a computer to operate pending AR and VR devices
        \item Talk to Professor Schrader about borrowing an old Opp Lab computer for development
        \item Talk to Dr. Rebenitsch about borrowing development work space 
    \end{itemize}

    \subsection{Successes}
    \label{sec:Sprint2_successes}
        \begin{itemize}
            \item Completed the first client presentation
            \item Able to convert file types in a standalone application
            \item Simple file upload to the ASP.NET web root folder is successful
            \item The minimum specifications for a development computer have been defined - ~\autoref{table:metatwosystemrequirements}
        \end{itemize}

    \subsection{Failures}
    \label{sec:Sprint2_failures}
        \begin{itemize}
            \item Unable to borrow development workspace from Dr. Rebenitsch
            \item Professor Schrader has old computers that do not meet the hardware requirements
            \item Professor Schrader is unable to provide us with a development computer that is 
            property of SD Mines without a reserved space  
            \item Textures associated with Maplesoft files are not kept after conversion
            \item The initial proposed software contract presented to SD Mines has been rejected
        \end{itemize}

    \subsection{Deliverables}
    \label{sec:Sprint2_deliverables}
    \begin{itemize}
        \item The first client presentation
        \item Demonstrate that standalone file conversion is operable
        \item Demonstrate that simple files are able to be uploaded to the ASP.NET web root
    \end{itemize}

    \subsection{Sprint Review}
    \label{sec:Sprint2_review}
        \hspace{7mm}
        The proposal to Dr. Rebenitsch to borrow development space in the SD Mines mobile computing lab has
        been rejected so attaining a private development space is still on the backlog moving forward.  Until
        then, Professor Schrader was unable to supply the team with a development computer, so long as the
        computer is property of SD Mines.\\

        Whether a development computer is or is not able to be issued, any of the old Opp Lab computers that Could
        be supplied by Professor Schrader do not meet the minimum hardware specifications for operating AR or VR
        devices.\\

        Through testing the standalone file conversion application, it was learned that Maplesoft files are unable
        to export texture or color properties upon conversion.\\

        The last and potentially largest roadblock encountered throughout this sprint development cycle is the 
        update that the initial software contract proposal that was presented to the Business Office of SD Mines
        has been rejected due to the language in the document specific to 
        \textit{InTouch L.L.C.} 
        retaining intillectual property for the project.  The contact at the SD Mines Business Office is adamant that 
        claims to intillectual property will not be negotiated.

    \subsection{Sprint Retrospective}
    \label{sec:Sprint2_retrospective}
        \hspace{7mm}
        The team is less than satisfied with the result of this sprint but optimistic looking forward.  The
        recorded failing items for this sprint are attributed to outside forces that are not related to the 
        development process.\\

        On the subject of finding a private development workspace, the unanimous decision of the development team is
        that until one is able to be reserved, working sessions involving the AR and VR equipment that is hardware
        dependent will have to be scheduled with a team member who owns a laptop that meets the minimum hardware
        specifications.\\

        On the subject of minimum hardware specifications, the unanimous decision of the development team is that a 
        portion of the project's hardware budget can be set aside for purchasing the necessary upgrades to a computer
        provided by Professor Schrader.  However, the issue of purchasing additional hardware is to be tabled until a
        private development space is able to be reserved.\\

        On the subject of the Maplesoft conversion, research is being done into how to make Maplesoft correctly export
        model color and texture information when the file is converted to an intermediary file type.\\ 

        On the subject of the software contract negotiations with SD Mines, the unanimous decision of the development 
        team is to abandon reliance on the funding from the SD Mines Mobile Computing Grant.  AR hardware will instead 
        be purchased via 
        \textit{InTouch L.L.C.}
        to continue development until completion.  Students shall be compensated as developers following a
        license purchase of the completed product by SD Mines.