% !TEX encoding = UTF-8 Unicode
% !TEX root = DesignDocument.tex

\section{Mobile}
\subsection{Considerations}
\tab The following sections will cover detailed instructions for publishing an Android App from Android Studio. The first step is to build a "release" version of the app. Once this release app is published, users can install the app through the Google Play Store or directly on their devices. These steps are taken from \url{https://developer.android.com/studio/publish/index.html} where more detailed steps are available for publishing an app.

\subsection{Building Release App}
During development, Android Studio will build the app for "debug" but will need to be modified and rebuilt when it is time to release the app to users.
\subsubsection{Preparing App for Release}
There are multiple steps for preparing an app for release to clean up the code, optimize it, and ensure it is ready for publishing somewhere like the Google Play Store. These steps can be found in more detail at \url{https://developer.android.com/studio/publish/preparing.html}. Some of the major steps are listed below.
\begin{itemize}
    \item Turn off logging and debugging - remove Log calls in code and the android:debuggable attribute from the manifest file.
    \item Clean up project folders - remove unused files and any protected or proprietary files that may have been used in development. This includes assets.
    \item Remove libraries in /lib folder that may no longer be used in the program.
    \item Review manifest and build settings - specify permissions and version numbers and requirements.
    \item Address compatibility issues - different versions of Android and different screen sizes behave differently. Consider using support library to be available to more devices.
\end{itemize}
\subsubsection{Build Release APK}
\begin{itemize}
    \item Build a release APK of the app - in Android Studio under Build -> Build Variant, you want to select "release" to generate a release build to distribute.
    \item Test the release app - install the release build to a development device to ensure that it installs and runs properly.
\end{itemize} 

\subsection{Publishing and Distribution}
There are two main ways to distribute an Android app to users, either by publishing to Google Play or sharing the generated APK file.
\subsubsection{Publishing to Google Play}
There are three essential steps to publishing an app on Google Play, provided in the most up-to-date detail at \url{https://developer.android.com/studio/publish/index.html}. 
\begin{itemize}
    \item Prepare promotional materials - this includes screenshots, graphics, and promotional tet for the Play Store page.
    \item Configuring options and uploading assets - set details such as countries available, price, category, etc.
    \item Publish release version of application - upload the APK built previously to the Play Console and click "Publish" to make it official. Google Play console available at \url{https://developer.android.com/distribute/console/index.html}.
\end{itemize}
\subsubsection{Distributing APK}
Android apps can be distributed through email or websites as well, as long as users can access the APK file on their devices. If the release APK is sent through an email and viewed on the phone, they can tap on the file and it will allow them to install it directly. The same process works for downloading the file from a website or loading it into the file system from a computer. Either of these methods require the user to opt-in to installing apps from unknown sources. They need to enable installing unknown apps from their Settings app under Security -> Unknown Sources. After enabling this, they may install the apps directly from an email, website, or downloads app.