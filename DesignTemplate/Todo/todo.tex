
\chapter{Future Development Consideration}
\label{ch:TODO}

The Augmented Education platform is intended to be a two-year project that is developed by two separate computer science senior design groups. With the platform's foundation developed, the following includes lists of suggested future developments that have been proposed by the client, end users, and/or developers. These lists are may change depending on the future preferences of the client and the client's end users.

    \section{Website}
    \begin{itemize}
        \item Add TLS.

        \item Add password recovery with two-factor authentication.

        \item Add administrator account with administrator privileges over content, users, and groups.

        \item Add custom group option for sharing content. 

        \item Add social/collaboration space

        \item Add support for animated files. 

        \item Expand list of supported file types. 

        \item Expand list of supported hardware. 

    \end{itemize}

    \section{Mobile}
        \begin{enumerate}
            \item ARCore Android App
            \begin{itemize}
                \item Add support for embedded image files (PNG). Needs to be able to load in, and display, multiple images on top of each other at once.  The app currently can handle one, but some models (especially those made in Paint3D) need to have multiple displayed on top of each other to get the correct shading/color.

                \item Add filtering, ordering, and searching to the list of models.

                \item Add support for deleting models from the list.

                \item Add more controls to the AR viewer (rotate, drag, pinch to zoom)

                \item Search through the models directory and clean up the listing with models that do not exist.

                \item Add functionality to send a request to the website to get a displayable name from QR codes.

                \item Publish app in Google Play Store or provide other method for simple distribution.
 
                \item Web-Mobile endpoint to get printable name for a file to better display the models on mobile devices.

                \item Include descriptive fields for file information available on web. 

            \end{itemize}
            \item ARKit iOS App
            \begin{itemize}
                \item Determine 3D file type for compatibility with ARKit.

                \item Create basic app with transitions between screens.

                \item Add drawing functionality for a single model.

                \item Allow user to select model from a list on the phone.

                \item Add login from phone to website, with authentication and saved credentials.

                \item Retrieve file listing from website and download files as requested.

                \item Allow for QR code scanning to add models to the phone.

                \item Provide all existing functionality within the Android app that is available in iOS.

                \item Implement future additions to Android app (filtering, controls, etc.) to iOS app as well.

                \item Publish app in App Store for distribution.

            \end{itemize}
        \end{enumerate}

    \section{AR Headsets}

    Consider custom applications for AR headsets for certain academic departments. Features may include ability to iterate through stages of design, animation, real-time customization of 3D file in view space, components that may interact with other components in the view space, or multi-device view space. 