% !TEX root = DesignDocument.tex

\chapter{User Stories,  Requirements, and Product Backlog}
\section{Overview}


The overview should take the form of an executive summary.  Give the reader a feel 
for the purpose of the document, what is contained in the document, and an idea 
of the purpose for the system or product. 

 The user stories 
are provided by the stakeholders.  You will create he backlogs and the requirements, and document here.  
This chapter should contain 
details about each of the requirements and how the requirements are or will be 
satisfied in the design and implementation of the system.

Below:   list, describe, and define the requirements in this chapter.  
There could be any number of sub-sections to help provide the necessary level of 
detail. 




\section{User Stories}
This section can really be seen as the guts of the document.  This section should 
be the result of discussions with the stakeholders with regard to the actual functional 
requirements of the software.  It is the user stories that will be used in the 
work breakdown structure to build tasks to fill the product backlog for implementation 
through the sprints.

This section should contain sub-sections to define and potentially provide a breakdown 
of larger user stories into smaller user stories.   Each component must have a test identified, 
meaning you need to know how you plan to test it.  If a requirement is not testable, then 
some justification needs to be made on why the requirement has been included.  
 The results of the tests should go in the testing chapter. 



\subsection{Round Zero}

Main Goal:

View a Maple 3D model on a Microsoft Hololens

\subsubsection{AR Rendering}

\begin{itemize}
	\item As a faculty member, I want a Maple file to be automatically converted into an AR Tag on a cloud server.
	\item As a user, I want to be able to view an AR tag through a Microsoft Hololens to render a 3D model.
\end{itemize}

\subsubsection{Website Hosting}

\begin{itemize}
	\item As a faculty member, I want to upload a Maple 3D model to a cloud server.
	\item As a faculty member, I want to be able to download the AR tag for my document from a cloud server.
\end{itemize}

\subsubsection{Sprint Zero Breakdown}
User stories can be broken down into two main categories: AR Rendering and Website Hosting.  Half of the team will primarily work on 
the AR Rendering stories, and the other half the Website Hosting stories.  The main goal of these user stories is to view a
3D model from the Maple software on a Microsoft Hololens where the files are stored and managed in the cloud.

\subsection{Round One}

\subsubsection{AR Rendering}

\begin{itemize}
	\item As a user, I want to be able to view surface materials
	\item As a user, I want to be able to slice a 3D model and view a section of the model
	\item As a user, I want to be able to be able to switch between models quickly in the AR device
\end{itemize}

\subsubsection{Sprint Three Breakdown}

The clients shared some of their requests for how the files are rendered and viewed in an AR device.  These include viewing surfact

\subsection{User Story \#2} 

\subsubsection{User Story \#2 Breakdown}
User story \#2  .... 

\subsection{User Story \#3} 

\subsubsection{User Story \#3 Breakdown}
User story \#3  .... 



\section{Requirements and Design Constraints}
Use this section to discuss what requirements exist that deal with meeting the 
business need.  These requirements might equate to design constraints which can 
take the form of system, network, and/or user constraints.  Examples:  Windows 
Server only, iOS only, slow network constraints, or no offline, local storage capabilities. 


\subsection{System  Requirements}
%What are they? How will they impact the potential design? Are there alternatives

The basic system requirements to use the website are to have a web browser installed with internet access.  The user must have access to modeling software or a method to create/provide 3D models to the website.

In order to fully use the product, a user must have an Augmented Reality device.  Each device may have different system requirements.

For example, the Meta 2 requires a seperate computer in order to run.  The minimum and recommended specifications are listed below.  The list was crated in November 2017.

\begin{center}
	\begin{tabular}{ | c | c | c | }
		\hline
		& Minimum & Recommended \\ \hline
		OS & Windows 10 (64 bit) & 	Windows 10 (64 bit) \\ \hline
		CPU & Intel i7-4770 & Intel i7-6700 \\ \hline
		RAM & 8GB DDR3 & 16GB DDR4 \\ \hline
		GPU & NVIDIA GTX 960 & NVIDIA GTX 970 \\ \hline
		Hard Drive & 2GB Free Space & 2GB+ Free Space \\ \hline
		I/O Ports & 1X HDMI 1.4b and 2X USB 3.0 ports & 1X HDMI 1.4b and 2X USB 3.0 ports \\ \hline
		3D Engine & Unity 5.6 or higher & Unity 5.6 or higher \\ \hline
	\end{tabular}
	\\
	More up to date requirements can be found on the Meta 2 website at: \url{https://buy.metavision.com/}
\end{center}


\subsection{Network Requirements}
What are they? 


\subsection{Development Environment Requirements}
What are they?  Is the system supposed to be cross-platform? 

\subsection{Project  Management Methodology}
The stakeholders might restrict how the project implementation will be managed. 
 There may be constraints on when design meetings will take place.  There might 
be restrictions on how often progress reports need to be provided and to whom. 


\section{Specifications}
Any specifications that need to be understood?  Put it here.  

\section{Product Backlog}
The full initial product backlog should go here.  The sprint backlogs are located in the prototypes chapter.

 
\begin{itemize}
\item What system will be used to keep track of the backlogs and sprint status?
\item Will all parties have access to the Sprint and Product Backlogs?
\item How many Sprints will encompass this particular project?
\item How long are the Sprint Cycles?
\item Are there restrictions on source control? 
\end{itemize}


\section{Research or Proof of Concept Results}
This section is reserved for the discussion centered on any research that needed 
to take place before full system design.  The research efforts may have led to 
the need to actually provide a proof of concept for approval by the stakeholders. 
 The proof of concept might even go to the extent of a user interface design or 
mockups. 


\section{Supporting Material}


This document might contain references or supporting material which should be documented 
and discussed  either here if appropriate or more often in the appendices at the end.  This material may have been provided by the stakeholders  
or it may be material garnered from research tasks.

