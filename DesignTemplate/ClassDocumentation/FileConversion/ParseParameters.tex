\hypertarget{fileconversion_parseparameters}
{
    \label{fileconversion_parseparameters}
    \index{FileConversion@ParseParameters}
}

\subsection{ParseParameters}
    \paragraph{}
        Parses the parameters for the input and output file names.  It will set a flag to denote the success of the parsing.

    \subsubsection{Member Functions}

        \begin{itemize}
            \item ParseParameters(int argc, char ** argv)
            \item ~ParseParameters()
        \end{itemize}

        \paragraph{ParseParameters(int argc, char ** argv)}
            \hfill \break
            The constructor for the class.  It performs the parsing of the parameters.   
        
        \paragraph{~ParseParameters()}
            \hfill \break
            The destructor for the class.

    \subsubsection{Member Variables}
        
        \begin{itemize}
            \item (bool) success
            \item (string) inputFile
            \item (string) outputFile
            \item (string) fileExtention
        \end{itemize}

        \paragraph{(bool) success}
            A flag for whether a the parameter parsing was successful or not.
            \begin{itemize}
                \item True = Successful
                \item False = Unsuccessful
            \end{itemize}

        \paragraph{(string) inputFile}
            The path of the file to read for conversion.  This can be a relative path or a direct path.
        
        \paragraph{(string) outputFile}
            The path of where to write the converted file.  This can be a relative or direct path.
        
        \paragraph{(string) fileType}
            The type set with the -t option.  Used when creating the output file name if a file extention was not provided.
            